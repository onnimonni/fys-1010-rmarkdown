\documentclass[]{article}
\newcommand*{\authorfont}{\fontfamily{phv}\selectfont}

\usepackage[T1]{fontenc}
\usepackage[utf8]{inputenc}
\usepackage[finnish]{babel}                 % Suomenkielinen tavutus ja otsikot
\usepackage[left=34mm,right=34mm]{geometry} % Näillä säädät kätevästi sivun marginaalit
\usepackage{pslatex}                        % makes the entire document to use ps fonts => pdf looks better

%\usepackage[dvips]{graphicx}
\usepackage{psfrag}
\usepackage{array}
\usepackage{longtable}
\usepackage{amsmath}                        % kaavojen lisäominaisuuksia
\usepackage{amsfonts}
\usepackage{amssymb}
\usepackage{mathrsfs}
\usepackage{tabularx}
\usepackage{epsfig}
\usepackage{color}
\usepackage{verbatim}
\usepackage{rotating}
\usepackage{multirow}
\usepackage{enumerate}
\usepackage{layout}
\usepackage{cite}
\usepackage{wrapfig}                       % Tällä saa tekstin kiertämään kuvan tarvittaessa
\usepackage[tight]{subfigure}

% Header ja footer
\usepackage{fancyhdr}                       % ylä- ja alatunnisteet tarvittaessa (fancy headers)
\pagestyle{fancy}

\usepackage{abstract}
\renewcommand{\abstractname}{}    % clear the title
\renewcommand{\absnamepos}{empty} % originally center

\renewenvironment{abstract}
 {{%
    \setlength{\leftmargin}{0mm}
    \setlength{\rightmargin}{\leftmargin}%
  }%
  \relax}
 {\endlist}

\makeatletter
\def\@maketitle{%
  \newpage
%  \null
%  \vskip 2em%
%  \begin{center}%
  \let \footnote \thanks
    {\fontsize{18}{20}\selectfont\raggedright  \setlength{\parindent}{0pt} \@title \par}%
}
%\fi
\makeatother



\setcounter{secnumdepth}{0}

\usepackage{longtable,booktabs}


% Change font family
%\renewcommand*\rmdefault{ppl}

% Add tightlist so that pandoc can convert lists to this latex template
\providecommand{\tightlist}{%
  \setlength{\itemsep}{0pt}\setlength{\parskip}{0pt}}

\title{2.1 Vauhtipyörä  }



\author{\Large Onni Hakala\vspace{0.05in} \newline\normalsize\emph{218486} \newline\normalsize\emph{Tietotekniikka}   \and \Large Pyry Vehmas\vspace{0.05in} \newline\normalsize\emph{242160} \newline\normalsize\emph{Automaatiotekniikka}  }


\date{}

\usepackage{titlesec}

\titleformat*{\section}{\normalsize\bfseries}
\titleformat*{\subsection}{\normalsize\bfseries}
\titleformat*{\subsubsection}{\normalsize\bfseries}
\titleformat*{\paragraph}{\normalsize\itshape}
\titleformat*{\subparagraph}{\normalsize\itshape}

% Start new page after every section
\newcommand{\sectionbreak}{\clearpage}

\makeatletter
\@ifpackageloaded{hyperref}{}{%
\ifxetex
  \usepackage[setpagesize=false, % page size defined by xetex
              unicode=false, % unicode breaks when used with xetex
              xetex]{hyperref}
\else
  \usepackage[unicode=true]{hyperref}
\fi
}
\@ifpackageloaded{color}{
    \PassOptionsToPackage{usenames,dvipsnames}{color}
}{%
    \usepackage[usenames,dvipsnames]{color}
}
\makeatother
\hypersetup{breaklinks=true,
            bookmarks=true,
            pdfauthor={Onni Hakala () and Pyry Vehmas ()},
            pdfkeywords = {},  
            pdftitle={2.1 Vauhtipyörä},
            colorlinks=true,
            citecolor=blue,
            urlcolor=blue,
            linkcolor=blue,
            pdfborder={0 0 0}}
\urlstyle{same}  % don't use monospace font for urls



\begin{document}

% \pagenumbering{arabic}% resets `page` counter to 1 
%
% \maketitle
\pagestyle{empty}


\begin{center}
    \bgroup
    \renewcommand{\arraystretch}{2.0}
    \begin{tabular}{ | p{2.0cm} | p{5cm} | p{7cm} |}
    \hline
    % Institute
     TTY  &
    % Course
     FYS-1010 Fysiikan työt I  &
    % Date
     17.04.2017  \\ \hline
    % Course assistant initials
     MA  &
    % Assignment
     2.1 Vauhtipyörä  &
    % Write all authors here
    Onni Hakala - \emph{\small 218486} - \emph{\small Tietotekniikka}   \par Pyry Vehmas - \emph{\small 242160} - \emph{\small Automaatiotekniikka}    \\
    \hline
    \end{tabular}
    \egroup
\end{center}
\newpage

\pagestyle{plain}                       % Sivun tyylin määrittäminen (ei ylätunnisteita, eikä muutakaan)
\pagenumbering{roman}                   % Sivunumeroinnin määrittäminen (Roomalaiset numerot alkuun)
\setcounter{secnumdepth}{-1}            % Tällä komennolla edes luvut (chapter) eivät numeroidu sisällyslyetteloon

\linespread{1.2}                        % Riviväli valinta sisällysluetteloa varten
\selectfont                             % Tämä komento vaaditaan jotta ylläoleva rivivälin valinta tulee voimaan

{
\hypersetup{linkcolor=black}
\setcounter{tocdepth}{2}
\tableofcontents
\newpage
}

\setcounter{secnumdepth}{2}            % Luvut ja alaluvut, chapters -> subsubsections, numeroituvat sisällyslyetteloon
\linespread{1.3}                       % Rivivälin valinta tekstiin
\selectfont                            % Tämä komento vaaditaan jotta ylläoleva rivivälin valinta tulee voimaan

\pagenumbering{arabic}                 % Arabialaiset sivunumerot varsinaiseen tekstiosaan


\vskip 6.5pt

\noindent  \section{Vauhtipyörä}\label{vauhtipyora}

Tämän työn tarkoituksena oli selvittää vauhtipyörän hitausmomentti
pyörimisakselinsa suhteen pyörimisliikkeessä, että heilahdusliikkeessä.
Lisäksi määritimme kyseessä olevan hitausmomentin myös laskennellasesti.

\section{Johdanto}\label{johdanto}

Hitausmomentti eli inertiamomentti (tunnus \(J\) tai \(I\)) vastaa
pyörivässä liikkeessä etenemisliikkeen massaa. Hitausmomentin
SI-järjestelmän mukainen yksikkö on \(kg\cdot m^2\) (kilogramma kertaa
metri toiseen). Mitä suurempi kappaleen hitausmomentti on, sitä suurempi
momentti vaaditaan, jotta kappale saadaan kiihtymään halutulla
kulmakiihtyvyydellä.\footnote{\url{https://fi.wikipedia.org/wiki/Hitausmomentti}}

\section{Teoria}\label{teoria}

\subsection{Hitausmomentti
pyörimisliikkeestä}\label{hitausmomentti-pyorimisliikkeesta}

Energiataseen kaava energiaperiaatteen mukaisesti:
\[ \tag{1} mgh = \frac{1}{2}I_c\omega^2 + \frac{1}{2}mv^2+F_{\mu}n_12\pi\rho \]
jossa \(m\) on punnuksen massa, \(g\) on putoamiskiihtyvyys, \(h\) on
punnuksen korkeus maasta, \(Ic\) on pyörimisliikkeen hitausmomentti,
\(\omega\) on kulmanopeus, \(v\) on ratanopeus, \(F_\mu\) on liikekitka,
\(n_1\) on vauhtipyörän pyörähdysten lukumäärä punnuksen pudotessa ja
\(\rho\) on akselin poikkileikkauksen säde.\\
Punnuksen pudottua maahan, kitkasta johtuen vauhtipyörän pyörimisenergia
muuttuu lämmöksi. Tämän aikana vauhtipyörä pyörii \(n_2\) kierrosta.

\[ \tag{2} \frac{1}{2}I_c\omega^2=F_\mu n_2 2\pi\rho \]

Yhdistämme yhtälöt ja saamme eliminoitua kitkan. Tämän jälkeen
sijoitetaan \(\omega = \frac{v}{r}\). Sijoitetaan \(v=2\frac{h}{t}\),
koska tasaisesti kiihtyvän liikkeen loppunopeus on kaksi kertaa
keskinopeus. Näiden operaatioiden jälkeen yhtälö saadaan muotoon:

\[ \label{hitausmomenttipyorii} \tag{3} I_c=mr^2(\frac{gt^2}{2h}-1)(\frac{n_2}{n_2+n_1})\]

\subsection{Hitausmomentti
heilahdusliikkeestä}\label{hitausmomentti-heilahdusliikkeesta}

Heilahdusjakson ajalle pätee kaava \(I_z=I_c+Mb2\), missä
\[ \tag{4} I_z = \frac{T^2Mgb}{4\pi^2} \]\\
Kaavaa käyttäen voimme johtaa hitausmomentin \(I_c\):
\[ \label{hitausmomenttiheiluri}\tag{5} I_c = \frac{T^2Mgb}{4\pi^2} - Mb^2\]

\subsection{Hitausmomentti
laskennallisesti}\label{hitausmomentti-laskennallisesti}

Hitausmomentti saadaan laskennallisesti kaavasta:
\[ \label{hitausmomentti3} \tag{6} I_c = \frac{1}{2}MR^2 \]

\section{Työn suoritus}\label{tyon-suoritus}

\subsection{Pyörimisliike}\label{pyorimisliike}

Mittasimme hitausmomentin pyörimisliikkeen avulla kiinnittämällä
vauhtipyörän statiiviin ja laitoimme narun kiinni punnukseen. Mittasimme
punnuksen massan digitaalisella vaa'alla. Valitsimme pudotuskorkeuden h
mahdollisimman korkeaksi \(h=43,5\pm0.1 cm\), jotta reaktioajasta
aiheutuva virhe olisi mahdollisimman pieni. Mittasimme sen rullamitalla.

Mittasimme sekuntikelloa käyttäen ajan \(t\), joka punnukselta kuluu
putoamisen aikana. Toistimme kokeen monta kertaa virheen
pienentämiseksi.\\
Narukehän halkaisija mitattiin työntömitalla.\\
Laskimme molemmat, kuinka monta kierrosta \(n_1\) vauhtipyörä pyörii
punnuksen putoamisen jälkeen.

\subsection{Heilahdusliike}\label{heilahdusliike}

Liikutimme vauhtipyörää alemmaksi statiivissa hitausmomentin laskemiseen
heilahdusliikeen jaksonajan avulla. Tällä saimme vähennettyä statiivin
resonointia ja värinää. Värinän vähentämiseksi käytimme
heilahdusliikkeeseen hyvin pientä heilahduskulmaa.\\
Mittasimme sekuntikellolla kymmenen heilahduksen kokonaisjaksonajan
\(10T\) vähentääksemme reaktionopeudesta aiheutuvaa virhettä. Jakamalla
kokonaisaika kymmenellä jaksonajalla saatiin yhden jaksonaika \(T\). Koe
toistettiin kahdeksan kertaa.

Mittasimme työntömitalla vauhtipyörän kiinnityskohdan etäisyyden \(b\)
vauhtipyörän keskipisteestä. Käytimme digitaalivaakaa massan \(M\)
mittaamiseen ja rullamittaa vauhtipyörän halkaisijan selvittämiseen.

\subsection{Laskennallinen määritys}\label{laskennallinen-maaritys}

Laskimme vauhtipyörän hitausmomentin kaavaa \(\eqref{hitausmomentti3}\).

\section{Mittaustulokset ja
havainnot}\label{mittaustulokset-ja-havainnot}

\subsection{Pyörimisliike}\label{pyorimisliike-1}

Mittasimme kaikki vakiot ohessa näkyvään taulukkoon.

\begin{longtable}[]{@{}lll@{}}
\toprule
Suure & Tulos & tarkkuus\tabularnewline
\midrule
\endhead
\(h/cm\) & 42,7 & \(\pm0.1\)\tabularnewline
\(n_1/kpl\) & 2,5 & \(\pm1/12\)\tabularnewline
\(m/g\) & 95,7 & \(\pm0.1\)\tabularnewline
\(2r/mm\) & 54,1 & \(\pm0.1\)\tabularnewline
\bottomrule
\end{longtable}

Mittaustuloksiin vaikutti se, että joissain mittauksissa vauhtipyörä
alkoi resonoida, joka aiheutti ylimääräistä kitkaa, jonka takia sen
vauhti hidastui. Mittauksissa \textbf{2-6} vauhtipyörän resonointi
vähensi kierroksia merkittävästi.

\begin{longtable}[]{@{}lll@{}}
\toprule
\(t\) & \(t/s\) & \(n_2/kpl\)\tabularnewline
\midrule
\endhead
1 & 2,22 & 12,5\tabularnewline
2 & 2,12 & 11,125\tabularnewline
3 & 2,10 & 10,125\tabularnewline
4 & 2,14 & 11,25\tabularnewline
5 & 2,13 & 9,625\tabularnewline
6 & 2,31 & 10,5\tabularnewline
tarkkuus & \(\pm0.1\) & \(\pm0.125\)\tabularnewline
\bottomrule
\end{longtable}

\newpage

\subsection{Heilahdusliike}\label{heilahdusliike-1}

Mittasimme heilahduskeskiön \(b\) mittaamalla reikien välisen minimi- ja
maksimietäisyyden. \(2R_{min}=8.75cm\) \(2R_{max}=9.91\) eli
\(b= \frac{2R_{min}+2R_{max}}{2\cdot2} \approx 9,33cm\).

\textbf{Mitatut vakiot:}

\begin{longtable}[]{@{}lll@{}}
\toprule
suure & tulos & tarkkuus\tabularnewline
\midrule
\endhead
\texttt{b/mm} & 93.3 & \(\pm0.1\)\tabularnewline
\texttt{M/g} & 801.0 & \(\pm0.1\)\tabularnewline
\texttt{2R/mm} & 197.0 & \(\pm0.3\)\tabularnewline
\bottomrule
\end{longtable}

\textbf{Mitatut jaksonajat:}

\begin{longtable}[]{@{}lll@{}}
\toprule
\(t\) & \(10 T/s\) & \(T/s\)\tabularnewline
\midrule
\endhead
1 & 8,12 & 0,812\tabularnewline
2 & 7,72 & 0,772\tabularnewline
3 & 7,72 & 0,772\tabularnewline
4 & 7,97 & 0.797\tabularnewline
5 & 7,87 & 0.787\tabularnewline
tarkkuus & \(\pm0.1\) & \(\pm0.01\)\tabularnewline
\bottomrule
\end{longtable}

\subsection{Laskennallinen määritys}\label{laskennallinen-maaritys-1}

Pyörän säteen \(R\) mittaamista varten mittasimme halkaisijan
\(2R=197mm=0.197m\) ja massan \(M=801g=0.801kg\)

\section{Tulosten laskenta}\label{tulosten-laskenta}

Mittaustuloksista laskettiin keskiarvot hitausmomentin laskemista
varten.

\subsection{Pyörimisliike}\label{pyorimisliike-2}

Punnuksen putoamiseen kuluvan ajan keskiarvoksi saatiin \(t=2.17s\) ja
vauhtipyörän kierrosten keskiarvoksi saatiin \(n_2=10.85\).\\
Hitausmomentin suuruudeksi pyörimisliikkeen avulla määritettynä saatiin
kaavan \(\eqref{hitausmomenttipyorii}\) avulla
\(I_c = 0,0048kg\cdot m^2\).

\subsection{Heilahdusliike}\label{heilahdusliike-2}

Heilahdusliikkeen heilahduksen jaksonajan keskiarvoksi saatiin
\(T=0.788s\).\\
Hitausmomentin suuruudeksi heilahdusliikkeen avulla määritettynä saatiin
kaavan \(\eqref{hitausmomenttiheiluri}\) avulla laskettuna
\(I_c = 0,00455kg\cdot m^2\).

\subsection{Laskennallinen määritys}\label{laskennallinen-maaritys-2}

Laskennallisesti määritettynä hitausmomentin suuruudeksi saatiin kaavan
\(\eqref{hitausmomentti3}\) avulla \(I_c = 0,0039kg\cdot m^2\).

\section{Yhteenveto}\label{yhteenveto}

Eri mittaustavoilla lasketuista hitausmomentin arvoista saadaan
suhteellisen samat arvot, eli mittauksissa on onnistuttu. Eniten muista
eroava tulos saatiin pyörimisliike mittauksella. Tämä selittynee
kuitenkin, sillä että havaitsimme mittauksen aikana resonointia mittaus
laitteessa, joka voi vaikuttaa tulokseen. Pääpiirteittäin mittauksissa
kuitenkin onnistuttiin hyvin ja saadut tulokset ovat haluttuja.

\section{Kirjallisuusviitteet}\label{kirjallisuusviitteet}

\begin{enumerate}
\def\labelenumi{\arabic{enumi}.}
\tightlist
\item
  \url{https://fi.wikipedia.org/wiki/Hitausmomentti}
\item
  Fysiikan työt I --työohje, 2.1 Vauhtipyörä {[}Moodle-oppimisalusta{]}.
  Saatavissa: \url{https://moodle2.tut.fi/} \textgreater{} FYS-1010
  \textgreater{} Aineistot \textgreater{} Työohjeet \textgreater{}
  Vauhtipyörä.
\item
  Fysiikan työt I --opintomoniste, 2.1 Vauhtipyörä
  {[}Moodle-oppimisalusta{]}. Saatavissa: \url{https://moodle2.tut.fi/}
  \textgreater{} FYS-1010 \textgreater{} Aineistot \textgreater{}
  Opintomoniste \textgreater{} Vauhtipyörä.
\end{enumerate}

\section{Liitteet}\label{liitteet}

\begin{enumerate}
\def\labelenumi{\arabic{enumi}.}
\tightlist
\item
  Mittauspyötäkirja
\end{enumerate}
\newpage 
\end{document}