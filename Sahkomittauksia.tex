\documentclass[]{article}
\newcommand*{\authorfont}{\fontfamily{phv}\selectfont}

\usepackage[T1]{fontenc}
\usepackage[utf8]{inputenc}
\usepackage[finnish]{babel}                 % Suomenkielinen tavutus ja otsikot
\usepackage[left=34mm,right=34mm]{geometry} % Näillä säädät kätevästi sivun marginaalit
\usepackage{pslatex}                        % makes the entire document to use ps fonts => pdf looks better

%\usepackage[dvips]{graphicx}
\usepackage{psfrag}
\usepackage{array}
\usepackage{longtable}
\usepackage{amsmath}                        % kaavojen lisäominaisuuksia
\usepackage{amsfonts}
\usepackage{amssymb}
\usepackage{mathrsfs}
\usepackage{tabularx}
\usepackage{epsfig}
\usepackage{color}
\usepackage{verbatim}
\usepackage{rotating}
\usepackage{multirow}
\usepackage{enumerate}
\usepackage{layout}
\usepackage{cite}
\usepackage{wrapfig}                       % Tällä saa tekstin kiertämään kuvan tarvittaessa
\usepackage[tight]{subfigure}

% Header ja footer
\usepackage{fancyhdr}                       % ylä- ja alatunnisteet tarvittaessa (fancy headers)
\pagestyle{fancy}

\usepackage{abstract}
\renewcommand{\abstractname}{}    % clear the title
\renewcommand{\absnamepos}{empty} % originally center

\renewenvironment{abstract}
 {{%
    \setlength{\leftmargin}{0mm}
    \setlength{\rightmargin}{\leftmargin}%
  }%
  \relax}
 {\endlist}

\makeatletter
\def\@maketitle{%
  \newpage
%  \null
%  \vskip 2em%
%  \begin{center}%
  \let \footnote \thanks
    {\fontsize{18}{20}\selectfont\raggedright  \setlength{\parindent}{0pt} \@title \par}%
}
%\fi
\makeatother



\setcounter{secnumdepth}{0}

\usepackage{color}
\usepackage{fancyvrb}
\newcommand{\VerbBar}{|}
\newcommand{\VERB}{\Verb[commandchars=\\\{\}]}
\DefineVerbatimEnvironment{Highlighting}{Verbatim}{commandchars=\\\{\}}
% Add ',fontsize=\small' for more characters per line
\usepackage{framed}
\definecolor{shadecolor}{RGB}{248,248,248}
\newenvironment{Shaded}{\begin{snugshade}}{\end{snugshade}}
\newcommand{\KeywordTok}[1]{\textcolor[rgb]{0.13,0.29,0.53}{\textbf{#1}}}
\newcommand{\DataTypeTok}[1]{\textcolor[rgb]{0.13,0.29,0.53}{#1}}
\newcommand{\DecValTok}[1]{\textcolor[rgb]{0.00,0.00,0.81}{#1}}
\newcommand{\BaseNTok}[1]{\textcolor[rgb]{0.00,0.00,0.81}{#1}}
\newcommand{\FloatTok}[1]{\textcolor[rgb]{0.00,0.00,0.81}{#1}}
\newcommand{\ConstantTok}[1]{\textcolor[rgb]{0.00,0.00,0.00}{#1}}
\newcommand{\CharTok}[1]{\textcolor[rgb]{0.31,0.60,0.02}{#1}}
\newcommand{\SpecialCharTok}[1]{\textcolor[rgb]{0.00,0.00,0.00}{#1}}
\newcommand{\StringTok}[1]{\textcolor[rgb]{0.31,0.60,0.02}{#1}}
\newcommand{\VerbatimStringTok}[1]{\textcolor[rgb]{0.31,0.60,0.02}{#1}}
\newcommand{\SpecialStringTok}[1]{\textcolor[rgb]{0.31,0.60,0.02}{#1}}
\newcommand{\ImportTok}[1]{#1}
\newcommand{\CommentTok}[1]{\textcolor[rgb]{0.56,0.35,0.01}{\textit{#1}}}
\newcommand{\DocumentationTok}[1]{\textcolor[rgb]{0.56,0.35,0.01}{\textbf{\textit{#1}}}}
\newcommand{\AnnotationTok}[1]{\textcolor[rgb]{0.56,0.35,0.01}{\textbf{\textit{#1}}}}
\newcommand{\CommentVarTok}[1]{\textcolor[rgb]{0.56,0.35,0.01}{\textbf{\textit{#1}}}}
\newcommand{\OtherTok}[1]{\textcolor[rgb]{0.56,0.35,0.01}{#1}}
\newcommand{\FunctionTok}[1]{\textcolor[rgb]{0.00,0.00,0.00}{#1}}
\newcommand{\VariableTok}[1]{\textcolor[rgb]{0.00,0.00,0.00}{#1}}
\newcommand{\ControlFlowTok}[1]{\textcolor[rgb]{0.13,0.29,0.53}{\textbf{#1}}}
\newcommand{\OperatorTok}[1]{\textcolor[rgb]{0.81,0.36,0.00}{\textbf{#1}}}
\newcommand{\BuiltInTok}[1]{#1}
\newcommand{\ExtensionTok}[1]{#1}
\newcommand{\PreprocessorTok}[1]{\textcolor[rgb]{0.56,0.35,0.01}{\textit{#1}}}
\newcommand{\AttributeTok}[1]{\textcolor[rgb]{0.77,0.63,0.00}{#1}}
\newcommand{\RegionMarkerTok}[1]{#1}
\newcommand{\InformationTok}[1]{\textcolor[rgb]{0.56,0.35,0.01}{\textbf{\textit{#1}}}}
\newcommand{\WarningTok}[1]{\textcolor[rgb]{0.56,0.35,0.01}{\textbf{\textit{#1}}}}
\newcommand{\AlertTok}[1]{\textcolor[rgb]{0.94,0.16,0.16}{#1}}
\newcommand{\ErrorTok}[1]{\textcolor[rgb]{0.64,0.00,0.00}{\textbf{#1}}}
\newcommand{\NormalTok}[1]{#1}

\usepackage{graphicx}
% We will generate all images so they have a width \maxwidth. This means
% that they will get their normal width if they fit onto the page, but
% are scaled down if they would overflow the margins.
\makeatletter
\def\maxwidth{\ifdim\Gin@nat@width>\linewidth\linewidth
\else\Gin@nat@width\fi}
\makeatother
\let\Oldincludegraphics\includegraphics
\renewcommand{\includegraphics}[1]{\Oldincludegraphics[width=\maxwidth]{#1}}

% Change font family
%\renewcommand*\rmdefault{ppl}

% Add tightlist so that pandoc can convert lists to this latex template
\providecommand{\tightlist}{%
  \setlength{\itemsep}{0pt}\setlength{\parskip}{0pt}}

\title{1.2 Sähkömittauksia  }



\author{\Large Onni Hakala\vspace{0.05in} \newline\normalsize\emph{218486} \newline\normalsize\emph{Tietotekniikka}   \and \Large Pyry Vehmas\vspace{0.05in} \newline\normalsize\emph{242160} \newline\normalsize\emph{Automaatiotekniikka}  }


\date{}

\usepackage{titlesec}

\titleformat*{\section}{\normalsize\bfseries}
\titleformat*{\subsection}{\normalsize\bfseries}
\titleformat*{\subsubsection}{\normalsize\bfseries}
\titleformat*{\paragraph}{\normalsize\itshape}
\titleformat*{\subparagraph}{\normalsize\itshape}

% Start new page after every section
\newcommand{\sectionbreak}{\clearpage}

\makeatletter
\@ifpackageloaded{hyperref}{}{%
\ifxetex
  \usepackage[setpagesize=false, % page size defined by xetex
              unicode=false, % unicode breaks when used with xetex
              xetex]{hyperref}
\else
  \usepackage[unicode=true]{hyperref}
\fi
}
\@ifpackageloaded{color}{
    \PassOptionsToPackage{usenames,dvipsnames}{color}
}{%
    \usepackage[usenames,dvipsnames]{color}
}
\makeatother
\hypersetup{breaklinks=true,
            bookmarks=true,
            pdfauthor={Onni Hakala () and Pyry Vehmas ()},
            pdfkeywords = {},  
            pdftitle={1.2 Sähkömittauksia},
            colorlinks=true,
            citecolor=blue,
            urlcolor=blue,
            linkcolor=blue,
            pdfborder={0 0 0}}
\urlstyle{same}  % don't use monospace font for urls



\begin{document}

% \pagenumbering{arabic}% resets `page` counter to 1 
%
% \maketitle
\pagestyle{empty}


\begin{center}
    \bgroup
    \renewcommand{\arraystretch}{2.0}
    \begin{tabular}{ | p{2.0cm} | p{5cm} | p{7cm} |}
    \hline
    % Institute
     TTY  &
    % Course
     FYS-1010 Fysiikan työt I  &
    % Date
     11.4.2017  \\ \hline
    % Course assistant initials
     MA  &
    % Assignment
     1.2 Sähkömittauksia  &
    % Write all authors here
    Onni Hakala - \emph{\small 218486} - \emph{\small Tietotekniikka}   \par Pyry Vehmas - \emph{\small 242160} - \emph{\small Automaatiotekniikka}    \\
    \hline
    \end{tabular}
    \egroup
\end{center}
\newpage

\pagestyle{plain}                       % Sivun tyylin määrittäminen (ei ylätunnisteita, eikä muutakaan)
\pagenumbering{roman}                   % Sivunumeroinnin määrittäminen (Roomalaiset numerot alkuun)
\setcounter{secnumdepth}{-1}            % Tällä komennolla edes luvut (chapter) eivät numeroidu sisällyslyetteloon

\linespread{1.2}                        % Riviväli valinta sisällysluetteloa varten
\selectfont                             % Tämä komento vaaditaan jotta ylläoleva rivivälin valinta tulee voimaan

{
\hypersetup{linkcolor=black}
\setcounter{tocdepth}{2}
\tableofcontents
\newpage
}

\setcounter{secnumdepth}{2}            % Luvut ja alaluvut, chapters -> subsubsections, numeroituvat sisällyslyetteloon
\linespread{1.3}                       % Rivivälin valinta tekstiin
\selectfont                            % Tämä komento vaaditaan jotta ylläoleva rivivälin valinta tulee voimaan

\pagenumbering{arabic}                 % Arabialaiset sivunumerot varsinaiseen tekstiosaan


\vskip 6.5pt

\noindent  Test test test!

\begin{Shaded}
\begin{Highlighting}[]
\CommentTok{# Fit regression line}
\KeywordTok{require}\NormalTok{(stats)}
\NormalTok{reg<-}\KeywordTok{lm}\NormalTok{(Virta }\OperatorTok{~}\StringTok{ }\NormalTok{Jännite, }\DataTypeTok{data =}\NormalTok{ r_1_v)}
\NormalTok{coeff=}\KeywordTok{coefficients}\NormalTok{(reg)}
\CommentTok{# Equation is straight line of format y = k*x}
\NormalTok{eq =}\StringTok{ }\KeywordTok{paste0}\NormalTok{(}\StringTok{"y = "}\NormalTok{, }\KeywordTok{round}\NormalTok{(coeff[}\DecValTok{2}\NormalTok{],}\DecValTok{1}\NormalTok{), }\StringTok{"*x "}\NormalTok{)}
\CommentTok{# Plot the data}
\KeywordTok{plot}\NormalTok{(r_1_v, }\DataTypeTok{main=}\NormalTok{eq)}
\KeywordTok{abline}\NormalTok{(reg, }\DataTypeTok{col=}\StringTok{"blue"}\NormalTok{)}
\end{Highlighting}
\end{Shaded}

\includegraphics{Sahkomittauksia_files/figure-latex/unnamed-chunk-2-1.pdf}
\newpage 
\end{document}